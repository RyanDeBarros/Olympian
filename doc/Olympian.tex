\documentclass[10pt]{report}
\usepackage[margin=1.0in]{geometry}
\usepackage{amsmath}
\usepackage{amssymb}
\usepackage{xcolor}
\usepackage{framed}
\usepackage{amsthm}
\usepackage{braket}

\newcommand{\floor}[1]{\left\lfloor #1 \right\rfloor}
\newcommand{\ceil}[1]{\left\lceil #1 \right\rceil}
\newcommand{\quarterturn}{\mathcal{R^*}}
\newcommand{\mat}[1]{\begin{pmatrix} #1 \end{pmatrix}}

\DeclareMathOperator{\clamp}{clamp}
\DeclareMathOperator{\chs}{choose}
\DeclareMathOperator{\avg}{avg}
\DeclareMathOperator*{\argmax}{arg\,max}
\DeclareMathOperator*{\argmin}{arg\,min}
\DeclareMathOperator{\eff}{eff}
\DeclareMathOperator{\rel}{rel}

\colorlet{shadecolor}{yellow!10}
\newtheorem{theorem}{Theorem}
\newtheorem{corollary}{Corollary}
\newenvironment{thm}
	{\begin{shaded}\begin{theorem}}
	{\end{theorem}\end{shaded}}
\newenvironment{coro}
	{\begin{shaded}\begin{corollary}}
	{\end{corollary}\end{shaded}}

\begin{document}
\title{Olympian Engine}
\author{Ryan de Barros}
\date{May 27, 2025}
\maketitle

\tableofcontents

\chapter{Engine Layout}

\chapter{Collision Testing}

\section{Introduction}
The MTV (minimum translation vector), or impulse, is represented by $\vec{\zeta}$. A contact point for shape $S_i$ is represented by $\vec{\chi}_i$. The normalization of a vector $\vec{v}$, $\hat{v}=\frac{1}{|\vec{v}|}\vec{v}$, is equivalently represented by $\braket{\vec{v}}$ when dealing with more verbose expressions. The quarter-turn (90-degree CCW rotation) of a vector $\vec{v}$ is denoted by $\quarterturn\vec{v}$.

\section{The Separating Axis Theroem (SAT)}

\section{The Gilbert-Johnson-Keerthi (GJK) algorithm}

\section{Primitives}

\subsection{Axis-Aligned Bounding Box (AABB)}

\subsection{Oriented Bounding Box (OBB)}

\subsection{k-Discrete Oriented Polytope (kDOP)}

\subsection{Convex Hull}

\subsection{Affine Circle}
Circles would indeed be the simplest primitives to work with if it weren't for the case of Affine Circles, circles that undergo some global affine transformation. Recall that this transformation can be represented by $G(\vec{v}_{\mbox{point}})=L\vec{v}_{\mbox{point}}+\vec{T}$ and $G(\vec{v}_{\mbox{ray}})=L\vec{v}_{\mbox{ray}}$. Note that in general, an Affine Circle is \textit{not} necessarily an ellipse.

\subsubsection{Fast-wrap algorithm}

\subsubsection{Deepest point algorithm}

\subsubsection{Non-affine Circle-Circle overlap}
This is the simplest possible test. Given two non-affine circles, parametrized by $(\vec{c}_1,r_1)$ and $(\vec{c}_2,r_2)$, the overlap test is simply
\begin{equation}|\vec{c_2}-\vec{c_1}|\leq r_1+r_2\end{equation}

\subsubsection{Non-affine Circle-Circle collision}
If two non-affine circles overlap, then the mtv is calculated as:
\begin{equation}\vec{\zeta}=(r_1+r_2-|\vec{c_1}-\vec{c_2}|)\braket{\vec{c_1}-\vec{c_2}}\end{equation}

\subsubsection{Non-affine Circle-Circle contact}
If two non-affine circles collide, then the contact point on $\vec{c_1}$ is:
\begin{equation}\vec{\chi}_1=\vec{c_1}+r_1\braket{\vec{\zeta}}\end{equation}

\subsubsection{Affine Circle-Circle overlap}
Unfortunately, an Affine Circle may be worse than an ellipse, which already requires solving a complex quartic equation when testing for collision with another ellipse or circle. Fortunately, there is a clearer and equally efficient golden-search algorithm that allows for approximation of the MTV. If a negative MTV occurs, the algorithm can be left early with a no-collision result. For overlapping, simply check that the MTV is non-negative, the approximation of which is described in the next section.

\subsubsection{Affine Circle-Circle collision}

\subsubsection{Affine Circle-Circle contact}

\subsubsection{General Circle-Polygon overlap}

\subsubsection{General Circle-Polygon collision}

\subsubsection{General Circle-Polygon contact}

\section{Compounds}

\section{Bounding Volume Hierarchies}

\section{Dynamics}

The physics variables of a rigid body can be organized into 4 categories:
\begin{itemize}
\item \textbf{State} (variable data that's updated every frame)
\begin{itemize}
\item Position ($\boldsymbol{x}$)
\item Rotation ($\theta$)
\item Linear velocity ($\boldsymbol{v}$)
\item Angular velocity ($\omega$)
\end{itemize}
\item \textbf{Properties} (stimuli applied to a particular instance)
\begin{itemize}
\item Linear acceleration ($\boldsymbol{a}$)
\item Angular acceleration ($\alpha$)
\item Force ($\boldsymbol{F}$) with optional local contact point ($\boldsymbol{\chi}_{\boldsymbol{F}}$)
\item Torque ($\tau$)
\item Linear impulse ($\boldsymbol{J}$) with optional local contact point ($\boldsymbol{\chi}_{\boldsymbol{J}}$)
\item Angular impulse ($H$)
\item Mass ($m\geq0$)
\item Moment of inertia ($I\geq0$)
\end{itemize}
\item \textbf{Material} (how a generic rigid body responds to stimuli)
\begin{itemize}
\item Restitution ($\varepsilon\in[0,1]$)
\item Linear drag ($\delta\geq0$)
\item Angular drag ($\sigma\geq0$)
\item Friction ($\mu\geq0$)
\begin{itemize}
\item 0 if body is not colliding with anything
\item Static friction ($\mu_s$) if body is stationary
\item Kinetic friction ($\mu_k$) if body is sliding
\item Rolling friction ($\mu_r$) if body is rolling
\end{itemize}
\item Resolution bias ($\Phi\in[0,1]$)
\end{itemize}
\item \textbf{Collision response}
\begin{itemize}
\item Minimum translation vector $\boldsymbol{\zeta}$
\item Contact point $\boldsymbol{\chi}$
\end{itemize}
\end{itemize}
Alternatively to storing mass in properties, one could store volume (area) instead, and density in material, so that mass can be computed generically. The resolution process is as follows:
\begin{enumerate}
\item Compute total linear acceleration $\boldsymbol{a}_\Sigma$.
\item Compute total angular acceleraiton $\alpha_\Sigma$.
\item $\Delta\boldsymbol{v}=\boldsymbol{a}_\Sigma\Delta t\implies\boldsymbol{v}^*=(\boldsymbol{v}^*/\boldsymbol{v})(\boldsymbol{v}+\Delta\boldsymbol{v})$
\item $\Delta\omega=\alpha_\Sigma\Delta t\implies\omega^*=(\omega^*/\omega)(\omega+\Delta\omega)$
\item $\Delta\boldsymbol{x}=\boldsymbol{v}\Delta t$
\item $\Delta\theta=\omega\Delta t$
\end{enumerate}

\subsection{Dynamics contributions for single collision}

\subsubsection{Standard accelerations}
\begin{equation}\Delta\boldsymbol{v}_{\boldsymbol{F}}=\int_t^{t+\Delta t}\boldsymbol{a}_{\boldsymbol{F}}dt=\int_t^{t+\Delta t}\frac{1}{m}\boldsymbol{F}dt=\frac{\Delta t}{m}\boldsymbol{F}\end{equation}
\begin{equation}\Delta\boldsymbol{v}_{\boldsymbol{J}}=\int_t^{t+\Delta t}\boldsymbol{a}_{\boldsymbol{J}}dt=\int_t^{t+\Delta t}\frac{1}{m}\boldsymbol{J}\delta(t-t^*)dt=\frac{1}{m}\boldsymbol{J}\end{equation}
\begin{equation}\Delta\omega_\tau=\int_t^{t+\Delta t}\alpha_\tau dt=\int_t^{t+\Delta t}\frac{1}{I}\tau dt=\frac{\tau\Delta t}{I}\end{equation}
\begin{equation}\Delta\omega_H=\int_t^{t+\Delta t}\alpha_Hdt=\int_t^{t+\Delta t}\frac{1}{I}H\delta(t-t^*)dt=\frac{H}{I}\end{equation}
\begin{equation}\Delta\omega_{\boldsymbol{\chi}}=\int_t^{t+\Delta t}\alpha_{\boldsymbol{\chi}}dt=\frac{\boldsymbol{\chi_{\boldsymbol{F}}}\times{\boldsymbol{F}}\Delta t}{I}+\frac{\boldsymbol{\chi_{\boldsymbol{J}}}\times{\boldsymbol{J}}}{I}\end{equation}

\subsubsection{Corrective teleportation due to collision}
If the resolution bias is 0, the position should be overridden via:
\begin{equation}\Delta\boldsymbol{x}=\Delta\boldsymbol{x}_T:=\boldsymbol{\zeta}\end{equation}
 The rotation should thus be overriden via:
\begin{equation}\Delta\boldsymbol{v}=\frac{1}{\Delta t}\boldsymbol{\zeta}\implies \boldsymbol{J}=m\Delta\boldsymbol{v}=\frac{m}{\Delta t}\boldsymbol{\zeta}\implies H=\boldsymbol{\chi}\times\boldsymbol{J}=\frac{m}{\Delta t}\boldsymbol{\chi}\times\boldsymbol{\zeta}\end{equation}
\begin{equation}\implies\Delta\omega=\frac{H}{I}=\frac{m}{I\Delta t}\boldsymbol{\chi}\times\boldsymbol{\zeta}\implies\Delta\theta=\Delta\theta_T:=\frac{m}{I}\boldsymbol{\chi}\times\boldsymbol{\zeta}\end{equation}
However, the above impulses are never used for updating linear/angular velocity. Additionally, to prevent overcorrection, the teleportation vector must be shared across both bodies:
\begin{equation}\Delta\boldsymbol{x}_{T,A}-\Delta\boldsymbol{x}_{T,B}=\boldsymbol{\zeta}\end{equation}
Given that both displacements are along the normal,
\begin{equation}\Delta\boldsymbol{x}_{T,A}=\eta_A\boldsymbol{\zeta},\quad\Delta\boldsymbol{x}_{T,B}=-\eta_B\boldsymbol{\zeta},\quad\eta\in[0,1]\end{equation}
\begin{equation}\implies\eta_A\boldsymbol{\zeta}+\eta_B\boldsymbol{\zeta}=\boldsymbol{\zeta}\implies\eta_A+\eta_B=1\end{equation}
The conservation of linear momentum over the frame is:
\begin{equation}m_A\Delta\boldsymbol{v}_A+m_B\Delta\boldsymbol{v}_B=\boldsymbol{0}\end{equation}
By assuming the correction begins with a frame where both bodies are stationary:
\begin{equation}m_A\boldsymbol{v}_A^*+m_B\boldsymbol{v}_B^*=\boldsymbol{0}\end{equation}
\begin{equation}m_A\frac{\Delta\boldsymbol{x}_{T,A}}{\Delta t}+m_B\frac{\Delta\boldsymbol{x}_{T,B}}{\Delta t}=\boldsymbol{0}\end{equation}
\begin{equation}m_A\Delta\boldsymbol{x}_{T,A}+m_B\Delta\boldsymbol{x}_{T,B}=\boldsymbol{0}\end{equation}
\begin{equation}m_A\eta_A\boldsymbol{\zeta}-m_B\eta_B\boldsymbol{\zeta}=\boldsymbol{0}\end{equation}
\begin{equation}m_A\eta_A=m_B\eta_B\end{equation}
Plugging back into the prior result:
\begin{equation}m_B\eta_A+m_A\eta_A=m_B\end{equation}
\begin{equation}\eta_A=\frac{m_B}{m_A+m_B}\end{equation}
Therefore, the actual teleportation is:
\begin{equation}\Delta\boldsymbol{x}_{T,A}=\frac{m_B}{m_A+m_B}\boldsymbol{\zeta}\end{equation}
\begin{equation}\Delta\theta_{T,A}=\frac{m_Am_B}{I_A(m_A+m_B)}\boldsymbol{\chi}_A\times\boldsymbol{\zeta}\end{equation}
Against a static body, the actual teleportation is the original one derived.

\subsubsection{Restitution contribution}
During a collision, there is a restitution impulse applied equally against the two objects. The restitution of the collision can be computed via several strategies:
\begin{itemize}
\item $\varepsilon=\min(\varepsilon_A,\varepsilon_B)$ (recommended)
\item $\varepsilon=\sqrt{\varepsilon_A\varepsilon_B}$
\item $\varepsilon=\frac{1}{2}(\varepsilon_A+\varepsilon_B)$
\item $\varepsilon=\varepsilon_A$
\end{itemize}
Define the relative velocity at a contact point as:
\begin{equation}\boldsymbol{v}_{AB,\boldsymbol{\chi}}=\boldsymbol{v}_A+\omega_A\boldsymbol{\chi}_A^\perp-\boldsymbol{v}_B-\omega_B\boldsymbol{\chi}_B^\perp\end{equation}
The impulse on A by B is:
\begin{equation}\boldsymbol{J}_R=j\braket{\boldsymbol{\zeta}}\end{equation}
Then by the conservation of linear momentum:
\begin{equation}\Delta\boldsymbol{v}_A=\frac{j}{m_A}\braket{\boldsymbol{\zeta}},\quad\Delta\boldsymbol{v}_B=-\frac{j}{m_B}\braket{\boldsymbol{\zeta}}\end{equation}
Thus, the angular change in velocity is:
\begin{equation}\Delta\omega_A=\frac{\boldsymbol{\chi}_A\times j\braket{\boldsymbol{\zeta}}}{I_A},\quad\Delta\omega_B=-\frac{\boldsymbol{\chi}_B\times j\braket{\boldsymbol{\zeta}}}{I_B}\end{equation}
Therefore:
\begin{equation}\Delta\boldsymbol{v}_{AB,\boldsymbol{\chi}}=\Delta\boldsymbol{v}_A+\Delta\omega_A\boldsymbol{\chi}_A^\perp-\Delta\boldsymbol{v}_B-\Delta\omega_B\boldsymbol{\chi}_B^\perp\end{equation}
\begin{equation}\Delta\boldsymbol{v}_{AB,\boldsymbol{\chi}}=\frac{j}{m_A}\braket{\boldsymbol{\zeta}}+\frac{\boldsymbol{\chi}_A\times j\braket{\boldsymbol{\zeta}}}{I_A}\boldsymbol{\chi}_A^\perp+\frac{j}{m_B}\braket{\boldsymbol{\zeta}}+\frac{\boldsymbol{\chi}_B\times j\braket{\boldsymbol{\zeta}}}{I_B}\boldsymbol{\chi}_B^\perp\end{equation}
\begin{equation}\Delta\boldsymbol{v}_{AB,\boldsymbol{\chi}}=j\left(\frac{1}{m_A}\braket{\boldsymbol{\zeta}}+\frac{\boldsymbol{\chi}_A\times\braket{\boldsymbol{\zeta}}}{I_A}\boldsymbol{\chi}_A^\perp+\frac{1}{m_B}\braket{\boldsymbol{\zeta}}+\frac{\boldsymbol{\chi}_B\times\braket{\boldsymbol{\zeta}}}{I_B}\boldsymbol{\chi}_B^\perp\right)\end{equation}
\begin{equation}\Delta\boldsymbol{v}_{AB,\boldsymbol{\chi}}\cdot\braket{\boldsymbol{\zeta}}=j\left(\frac{1}{m_A}+\frac{\left(\boldsymbol{\chi}_A\times\braket{\boldsymbol{\zeta}}\right)^2}{I_A}+\frac{1}{m_B}+\frac{\left(\boldsymbol{\chi}_B\times\braket{\boldsymbol{\zeta}}\right)^2}{I_B}\right)\end{equation}
\begin{equation}\Delta\boldsymbol{v}_{AB,\boldsymbol{\chi}}\cdot\braket{\boldsymbol{\zeta}}=\frac{1}{m_{\eff}}j\end{equation}
This is the definition of \textit{effective mass} at the contact point $\boldsymbol{\chi}$:
\begin{equation}m_{\eff}=\left(\frac{1}{m_A}+\frac{\left(\boldsymbol{\chi}_A\times\braket{\boldsymbol{\zeta}}\right)^2}{I_A}+\frac{1}{m_B}+\frac{\left(\boldsymbol{\chi}_B\times\braket{\boldsymbol{\zeta}}\right)^2}{I_B}\right)^{-1}\end{equation}
The definition for elasticity is:
\begin{equation}\varepsilon=-\frac{\boldsymbol{v}^*_{AB,\boldsymbol{\chi}}\cdot\boldsymbol{\zeta}}{\boldsymbol{v}_{AB,\boldsymbol{\chi}}\cdot\boldsymbol{\zeta}}\implies\boldsymbol{v}^*_{AB,\boldsymbol{\chi}}\cdot\boldsymbol{\zeta}=-\varepsilon\boldsymbol{v}_{AB,\boldsymbol{\chi}}\cdot\boldsymbol{\zeta}\end{equation}
\begin{equation}\boldsymbol{v}_{AB,\boldsymbol{\chi}}\cdot\boldsymbol{\zeta}+\Delta\boldsymbol{v}_{AB,\boldsymbol{\chi}}\cdot\boldsymbol{\zeta}=-\varepsilon\boldsymbol{v}_{AB,\boldsymbol{\chi}}\cdot\boldsymbol{\zeta}\implies\Delta\boldsymbol{v}_{AB,\boldsymbol{\chi}}\cdot\boldsymbol{\zeta}=-(1+\varepsilon)\boldsymbol{v}_{AB,\boldsymbol{\chi}}\cdot\boldsymbol{\zeta}\end{equation}
Plugging the effective mass in:
\begin{equation}\frac{j}{m_{\eff}}=-(1+\varepsilon)\boldsymbol{v}_{AB,\boldsymbol{\chi}}\cdot\braket{\boldsymbol{\zeta}}\implies j=\max(-m_{\eff}(1+\varepsilon)\boldsymbol{v}_{AB,\boldsymbol{\chi}}\cdot\braket{\boldsymbol{\zeta}}, 0)\end{equation}
\begin{equation}\boldsymbol{J}_R=\max\left(-m_{\eff}(1+\varepsilon)\left(\boldsymbol{v}_{AB,\boldsymbol{\chi}}\cdot\braket{\boldsymbol{\zeta}}\right), 0\right)\braket{\boldsymbol{\zeta}}\end{equation}
As for the angular impulse:
\begin{equation}H_R=\boldsymbol{\chi}\times\boldsymbol{J}_R\end{equation}

\subsubsection{Restitution contribution against a static object}
Suppose $B$ is a static body, i.e. $m_B=\infty$, $I_B=\infty$, $\boldsymbol{v_B}=\boldsymbol{0}$, and $\omega_B=0$. Then:
\begin{equation}\boldsymbol{v}_{AB,\boldsymbol{\chi}}=\boldsymbol{v}_A+\omega_A\boldsymbol{\chi}_A^\perp\end{equation}
\begin{equation}m_{\eff}=\left(\frac{1}{m_A}+\frac{\left(\boldsymbol{\chi}_A\times\braket{\boldsymbol{\zeta}}\right)^2}{I_A}\right)^{-1}=\frac{m_AI_A}{I_A+m_A\left(\boldsymbol{\chi}_A\times\braket{\boldsymbol{\zeta}}\right)^2}\end{equation}
\begin{equation}\boldsymbol{J}_R=-\frac{m_AI_A(1+\varepsilon)\left(\boldsymbol{v}_A\cdot\braket{\boldsymbol{\zeta}}+\omega_A\boldsymbol{\chi}_A\times\braket{\boldsymbol{\zeta}}\right)}{I_A+m_A\left(\boldsymbol{\chi}_A\times\braket{\boldsymbol{\zeta}}\right)^2}\braket{\boldsymbol{\zeta}}\end{equation}

\subsubsection{Friction contribution}
The restitution impulse acts as an instantaneous normal force, which causes friction.
\begin{equation}\|\boldsymbol{J}_f\|\leq\mu\|\boldsymbol{N}\Delta t\|=\mu\left\|\boldsymbol{J}_R\right\|\end{equation}
Define the following notation:
\begin{equation}\boldsymbol{T}_{\boldsymbol{u}}(\boldsymbol{v})=\boldsymbol{v}-(\boldsymbol{v}\cdot\braket{\boldsymbol{u}})\braket{\boldsymbol{u}}\end{equation}
Then, if $\boldsymbol{T}_{\boldsymbol{\zeta}}(\boldsymbol{v}_{AB,\boldsymbol{\chi}})\approx\boldsymbol{0}$, static or rolling friction will apply. Otherwise kinetic friction applies. In either case:
\begin{equation}\braket{\boldsymbol{J}_f}=-\braket{\boldsymbol{T}_{\boldsymbol{\zeta}}(\boldsymbol{v}_{AB,\boldsymbol{\chi}}^*)}\end{equation}
Where the (*) velocity is determined by non-friction forces. But the body cannot move in the direction of friction. In fact, if $\boldsymbol{T}_{\boldsymbol{\zeta}}(\boldsymbol{v}_{AB,\boldsymbol{\chi}}^*)\approx\boldsymbol{0}$, then $\boldsymbol{J}_f=\boldsymbol{0}$. Otherwise,
\begin{equation}\frac{1}{m_{\eff}}\|\boldsymbol{J}_f\|\leq\|\boldsymbol{T}_{\boldsymbol{\zeta}}(\boldsymbol{v}_{AB,\boldsymbol{\chi}}^*)\|\end{equation}
\begin{equation}\implies\|\boldsymbol{J}_f\|=\min(\mu\|\boldsymbol{J}_R\|, m_{\eff}\|\boldsymbol{T}_{\boldsymbol{\zeta}}(\boldsymbol{v}_{AB,\boldsymbol{\chi}}^*)\|)\end{equation}
Just like with restitution, combine $\mu_A$ with $\mu_B$ ($\mu=\sqrt{\mu_A\mu_B}$ is the recommended strategy). There is also an angular impulse contribution:
\begin{equation}H_f=\boldsymbol{\chi}\times\boldsymbol{J}_f\end{equation}

\subsubsection{Full impact response}
The full impact response is the combination of restitution impulse and friction impulse:
\begin{equation}\boldsymbol{J}_C=\boldsymbol{J}_R+\boldsymbol{J}_f\end{equation}
\begin{equation}H_C=H_R+H_f=\boldsymbol{\chi}\times\boldsymbol{J}_C\end{equation}
Note that this is used for positioning when resolution bias is 1. In full:
\begin{equation}\Delta\boldsymbol{x}=(1-\Phi)\Delta\boldsymbol{x}_T+\Phi\boldsymbol{v}\Delta t\end{equation}
\begin{equation}\Delta\theta=(1-\Phi)\Delta\theta_T+\Phi\omega\Delta t\end{equation}

\subsubsection{Full additive acceleration without collision}
\begin{equation}\Delta\boldsymbol{v}=\Delta\boldsymbol{v}_\Psi:=\sum_{\boldsymbol{a}}\boldsymbol{a}\Delta t+\frac{1}{m}\left(\sum_{\boldsymbol{F}}\boldsymbol{F}\Delta t+\sum_{\boldsymbol{J}}\boldsymbol{J}\right)\end{equation}
\begin{equation}\Delta\omega=\Delta\omega_\Psi:=\sum_\alpha\alpha\Delta t+\frac{1}{I}\left(m\sum_{\boldsymbol{a}}\boldsymbol{\chi}_{\boldsymbol{a}}\times\boldsymbol{a}\Delta t+\sum_\tau\tau\Delta t+\sum_HH+\sum_{\boldsymbol{F}}\boldsymbol{\chi_{\boldsymbol{F}}}\times{\boldsymbol{F}}\Delta t+\sum_{\boldsymbol{J}}\boldsymbol{\chi_{\boldsymbol{J}}}\times{\boldsymbol{J}}\right)\end{equation}

\subsubsection{Full additive acceleration with collision}
\begin{equation}\Delta\boldsymbol{v}=\Delta\boldsymbol{v}_\Psi+\frac{1}{m}\boldsymbol{J}_C\end{equation}
\begin{equation}\Delta\omega=\Delta\omega_\Psi+\frac{1}{I}\boldsymbol{\chi}\times\boldsymbol{J}_C\end{equation}

\subsubsection{Multiplicative acceleration due to drag}
\begin{equation}\frac{d\boldsymbol{v}}{dt}=-\delta\boldsymbol{v}\implies\boldsymbol{v}=\boldsymbol{C}e^{-\delta t}\end{equation}
Velocity on next frame ($\boldsymbol{v}^*$) due to drag is:
\begin{equation}\boldsymbol{v}^*=\boldsymbol{C}e^{-\delta(t+\Delta t)}=\boldsymbol{C}e^{-\delta t}e^{-\delta\Delta t}=\boldsymbol{v}e^{-\delta\Delta t}\end{equation}
\begin{equation}(\boldsymbol{v}^*/\boldsymbol{v})_\delta=e^{-\delta\Delta t}\end{equation}
Therefore, unlike adding $\Delta\boldsymbol{v}$ to velocity like with standard acceleration, multiply by $\boldsymbol{v}^*/\boldsymbol{v}$. Likewise;
\begin{equation}(\omega^*/\omega)_\sigma=e^{-\sigma\Delta t}\end{equation}

\subsection{Moment of Inertia}

\subsubsection{Affine circle}
\begin{equation}I=\frac{|L|\rho\pi r^4}{4}\mbox{tr}(L^TL)+|L|\rho\pi r^2\|L\boldsymbol{c}+\boldsymbol{T}\|^2=\frac{mr^2}{4}\mbox{tr}(L^TL)+m\|L\boldsymbol{c}+\boldsymbol{T}\|^2\end{equation}

\subsubsection{Polygon}
\begin{multline}I=\frac{\rho}{12}\sum_{i=1}^N(\boldsymbol{p}_i\times\boldsymbol{p}_{i+1})(\|\boldsymbol{p}_i\|^2+\boldsymbol{p}_i\cdot\boldsymbol{p}_{i+1}+\|\boldsymbol{p}_{i+1}\|^2)\\=\frac{m}{6\sum_{i=1}^N\boldsymbol{p}_i\times\boldsymbol{p}_{i+1}}\sum_{i=1}^N(\boldsymbol{p}_i\times\boldsymbol{p}_{i+1})(\|\boldsymbol{p}_i\|^2+\boldsymbol{p}_i\cdot\boldsymbol{p}_{i+1}+\|\boldsymbol{p}_{i+1}\|^2)\end{multline}

\subsubsection{Capsule}
A non-affine split circle has moment of inertia:
\begin{equation}I=\frac{1}{2}mr^2+mh^2\end{equation}
The affine split circle has moment of inertia:
\begin{equation}I=\frac{mr^2}{4}\mbox{tr}(L^TL)+\frac{1}{4}mh^2[L^TL]_{yy}+m\|L\boldsymbol{c}+\boldsymbol{T}\|^2\end{equation}
From now on, let $\boldsymbol{q}=L\boldsymbol{c}+\boldsymbol{T}$. The rectangular section has these coordinates:
\begin{itemize}
\item $\boldsymbol{p}_1=\frac{1}{2}L\mat{-w\\-h}+\boldsymbol{q}$
\item $\boldsymbol{p}_2=\frac{1}{2}L\mat{w\\-h}+\boldsymbol{q}$
\item $\boldsymbol{p}_3=\frac{1}{2}L\mat{w\\h}+\boldsymbol{q}$
\item $\boldsymbol{p}_4=\frac{1}{2}L\mat{-w\\h}+\boldsymbol{q}$
\end{itemize}
Therefore:
\begin{itemize}
\item $\|\boldsymbol{p}_1\|^2=\frac{1}{4}\left\|L\mat{-w\\-h}\right\|^2+L\mat{-w\\-h}\cdot\boldsymbol{q}+\|\boldsymbol{q}\|^2$
\item $\|\boldsymbol{p}_2\|^2=\frac{1}{4}\left\|L\mat{w\\-h}\right\|^2+L\mat{w\\-h}\cdot\boldsymbol{q}+\|\boldsymbol{q}\|^2$
\item $\|\boldsymbol{p}_3\|^2=\frac{1}{4}\left\|L\mat{w\\h}\right\|^2+L\mat{w\\h}\cdot\boldsymbol{q}+\|\boldsymbol{q}\|^2$
\item $\|\boldsymbol{p}_4\|^2=\frac{1}{4}\left\|L\mat{-w\\h}\right\|^2+L\mat{-w\\h}\cdot\boldsymbol{q}+\|\boldsymbol{q}\|^2$
\end{itemize}
\begin{itemize}
\item $\boldsymbol{p}_1\cdot\boldsymbol{p}_2=\frac{1}{4}L\mat{-w\\-h}\cdot L\mat{w\\-h}+L\mat{0\\-h}\cdot\boldsymbol{q}+\|\boldsymbol{q}\|^2$
\item $\boldsymbol{p}_2\cdot\boldsymbol{p}_3=\frac{1}{4}L\mat{w\\-h}\cdot L\mat{w\\h}+L\mat{w\\0}\cdot\boldsymbol{q}+\|\boldsymbol{q}\|^2$
\item $\boldsymbol{p}_3\cdot\boldsymbol{p}_4=\frac{1}{4}L\mat{w\\h}\cdot L\mat{-w\\h}+L\mat{0\\h}\cdot\boldsymbol{q}+\|\boldsymbol{q}\|^2$
\item $\boldsymbol{p}_4\cdot\boldsymbol{p}_1=\frac{1}{4}L\mat{-w\\h}\cdot L\mat{-w\\-h}+L\mat{-w\\0}\cdot\boldsymbol{q}+\|\boldsymbol{q}\|^2$
\end{itemize}
\begin{itemize}
\item $\|\boldsymbol{p}_1\|^2+\boldsymbol{p}_1\cdot\boldsymbol{p}_2+\|\boldsymbol{p}_2\|^2=h^2\|\boldsymbol{L}_2\|^2-\frac{1}{4}L\mat{-w\\-h}\cdot L\mat{w\\-h}-3h\boldsymbol{L}_2\cdot\boldsymbol{q}+3\|\boldsymbol{q}\|^2$
\item $\|\boldsymbol{p}_2\|^2+\boldsymbol{p}_2\cdot\boldsymbol{p}_3+\|\boldsymbol{p}_3\|^2=w^2\|\boldsymbol{L}_1\|^2-\frac{1}{4}L\mat{w\\-h}\cdot L\mat{w\\h}+3w\boldsymbol{L}_1\cdot\boldsymbol{q}+3\|\boldsymbol{q}\|^2$
\item $\|\boldsymbol{p}_3\|^2+\boldsymbol{p}_3\cdot\boldsymbol{p}_4+\|\boldsymbol{p}_4\|^2=h^2\|\boldsymbol{L}_2\|^2-\frac{1}{4}L\mat{w\\h}\cdot L\mat{-w\\h}+3h\boldsymbol{L}_2\cdot\boldsymbol{q}+3\|\boldsymbol{q}\|^2$
\item $\|\boldsymbol{p}_4\|^2+\boldsymbol{p}_4\cdot\boldsymbol{p}_1+\|\boldsymbol{p}_1\|^2=w^2\|\boldsymbol{L}_1\|^2-\frac{1}{4}L\mat{-w\\h}\cdot L\mat{-w\\-h}-3w\boldsymbol{L}_1\cdot\boldsymbol{q}+3\|\boldsymbol{q}\|^2$
\end{itemize}
\begin{itemize}
\item $\boldsymbol{p}_1\times\boldsymbol{p}_2=\frac{1}{2}wh|L|-w\boldsymbol{L}_1\times\boldsymbol{q}$
\item $\boldsymbol{p}_2\times\boldsymbol{p}_3=\frac{1}{2}wh|L|-h\boldsymbol{L}_2\times\boldsymbol{q}$
\item $\boldsymbol{p}_3\times\boldsymbol{p}_4=\frac{1}{2}wh|L|+w\boldsymbol{L}_1\times\boldsymbol{q}$
\item $\boldsymbol{p}_4\times\boldsymbol{p}_1=\frac{1}{2}wh|L|+h\boldsymbol{L}_2\times\boldsymbol{q}$
\end{itemize}
\begin{itemize}
\item $(\boldsymbol{p}_1\times\boldsymbol{p}_2)(\|\boldsymbol{p}_1\|^2+\boldsymbol{p}_1\cdot\boldsymbol{p}_2+\|\boldsymbol{p}_2\|^2)=\left(\frac{1}{2}wh|L|-w\boldsymbol{L}_1\times\boldsymbol{q}\right)\left(h^2\|\boldsymbol{L}_2\|^2-\frac{1}{4}L\mat{-w\\-h}\cdot L\mat{w\\-h}-3h\boldsymbol{L}_2\cdot\boldsymbol{q}+3\|\boldsymbol{q}\|^2\right)$
\item $(\boldsymbol{p}_2\times\boldsymbol{p}_3)(\|\boldsymbol{p}_2\|^2+\boldsymbol{p}_2\cdot\boldsymbol{p}_3+\|\boldsymbol{p}_3\|^2)=\left(\frac{1}{2}wh|L|-h\boldsymbol{L}_2\times\boldsymbol{q}\right)\left(w^2\|\boldsymbol{L}_1\|^2-\frac{1}{4}L\mat{w\\-h}\cdot L\mat{w\\h}+3w\boldsymbol{L}_1\cdot\boldsymbol{q}+3\|\boldsymbol{q}\|^2\right)$
\item $(\boldsymbol{p}_3\times\boldsymbol{p}_4)(\|\boldsymbol{p}_3\|^2+\boldsymbol{p}_3\cdot\boldsymbol{p}_4+\|\boldsymbol{p}_4\|^2)=\left(\frac{1}{2}wh|L|+w\boldsymbol{L}_1\times\boldsymbol{q}\right)\left(h^2\|\boldsymbol{L}_2\|^2-\frac{1}{4}L\mat{w\\h}\cdot L\mat{-w\\h}+3h\boldsymbol{L}_2\cdot\boldsymbol{q}+3\|\boldsymbol{q}\|^2\right)$
\item $(\boldsymbol{p}_4\times\boldsymbol{p}_1)(\|\boldsymbol{p}_4\|^2+\boldsymbol{p}_4\cdot\boldsymbol{p}_1+\|\boldsymbol{p}_1\|^2)=\left(\frac{1}{2}wh|L|+h\boldsymbol{L}_2\times\boldsymbol{q}\right)\left(w^2\|\boldsymbol{L}_1\|^2-\frac{1}{4}L\mat{-w\\h}\cdot L\mat{-w\\-h}-3w\boldsymbol{L}_1\cdot\boldsymbol{q}+3\|\boldsymbol{q}\|^2\right)$
\end{itemize}
So the moment of inertia of the rectangle is:
\begin{equation}
I=\frac{\rho wh|L|}{12}\left(w^2\|\boldsymbol{L}_1\|^2+h^2\|\boldsymbol{L}_2\|^2\right)+\rho wh|L|\|\boldsymbol{q}\|^2=\frac{m}{12}\left(w^2\|\boldsymbol{L}_1\|^2+h^2\|\boldsymbol{L}_2\|^2\right)+m\|\boldsymbol{q}\|^2
\end{equation}
In total, the moment of inertia of an initially upright capsule is:
\begin{equation}I=\frac{mr^2}{4}\mbox{tr}(L^TL)+\frac{mh^2}{4}[L^TL]_{yy}+\frac{m}{12}\left(w^2\|\boldsymbol{L}_1\|^2+h^2\|\boldsymbol{L}_2\|^2\right)+2m\|L\boldsymbol{c}+\boldsymbol{T}\|^2\end{equation}

\end{document}
