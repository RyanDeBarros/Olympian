\documentclass[10pt]{report}
\usepackage[margin=1.0in]{geometry}
\usepackage{amsmath}
\usepackage{amssymb}
\usepackage{xcolor}
\usepackage{framed}
\usepackage{amsthm}

\newcommand{\floor}[1]{\left\lfloor #1 \right\rfloor}
\newcommand{\ceil}[1]{\left\lceil #1 \right\rceil}
\DeclareMathOperator{\clamp}{clamp}
\DeclareMathOperator{\chs}{choose}

\colorlet{shadecolor}{yellow!10}
\newtheorem{theorem}{Theorem}
\newtheorem{corollary}{Corollary}
\newenvironment{thm}
	{\begin{shaded}\begin{theorem}}
	{\end{theorem}\end{shaded}}
\newenvironment{coro}
	{\begin{shaded}\begin{corollary}}
	{\end{corollary}\end{shaded}}

\begin{document}
\title{Olympian Engine}
\author{Ryan de Barros}
\date{Apr 9, 2025}
\maketitle

\tableofcontents

\chapter{Engine Layout}


\chapter{Rendering Design}

\section{Immutable batches}

\section{Mutable batches}

\section{Instanced renderering}


\chapter{Particle Systems}

\section{Introduction}

\subsection{Modular integration}
As a preface to all the math to come, here are a few formulas for modular integrals (note that $T\geq0$):

\begin{eqnarray}
\int_a^bf(\tau\bmod T)d\tau&=&\left(\int_{T\floor{a/T}}^{T(\floor{b/T}+1)}-\int_{T\floor{a/T}}^a-\int_b^{T(\floor{b/T}+1)}\right)f(\tau\bmod T)d\tau\\
&=&\sum_{k=\floor{a/T}}^{\floor{b/T}}\int_0^Tf(\tau\bmod T)d\tau-\left(\int_{T\floor{a/T}}^a+\int_b^{T(\floor{b/T}+1)}\right)f(\tau\bmod T)d\tau\\
&=&(\floor{b/T}-\floor{a/T}+1)\int_0^Tf(\tau)d\tau-\int_0^{a\bmod T}f(\tau)d\tau-\int_{b\bmod T}^Tf(\tau)d\tau\\
&=&\int_{a\bmod T}^{b\bmod T}f(\tau)d\tau+(\floor{\tfrac{b}{T}}-\floor{\tfrac{a}{T}})\int_0^Tf(\tau)d\tau
\end{eqnarray}

\begin{eqnarray}
\int_a^bf(\tau,\tau\bmod T)d\tau&=&\left(\int_{T\floor{a/T}}^{T(\floor{b/T}+1)}-\int_{T\floor{a/T}}^a-\int_b^{T(\floor{b/T}+1)}\right)f(\tau,\tau\bmod T)d\tau\\
&=&\sum_{k=0}^{\floor{b/T}-\floor{a/T}}\int_{T(\floor{a/T}+k)}^{T(\floor{a/T}+k+1)}f(\tau,\tau\bmod T)d\tau\nonumber\\&&-\left(\int_{T\floor{a/T}}^a+\int_b^{T(\floor{b/T}+1)}\right)f(\tau,\tau\bmod T)d\tau\\
&=&\int_0^T\sum_{k=0}^{\floor{b/T}-\floor{a/T}}f(\tau+T(\floor{\tfrac{a}{T}}+k),\tau)d\tau\nonumber\\&&-\int_0^{a\bmod T}f(\tau+T\floor{\tfrac{a}{T}},\tau)d\tau-\int_{b\bmod T}^Tf(\tau+T\floor{\tfrac{b}{T}},\tau)d\tau
\end{eqnarray}

\begin{eqnarray}
\int_a^bf(g(\tau)\bmod T)d\tau&=&\int_{g(a)}^{g(b)}f(u\bmod T)h'(u)du\qquad\mbox{where }h=g^{-1}\\
&=&\int_0^T\sum_{k=0}^{\floor{g(b)/T}-\floor{g(a)/T}}f(\tau)h'(\tau+T(\floor{g(a)/T}+k))d\tau\nonumber\\
&&-\int_0^{g(a)\bmod T}f(\tau)h'(\tau+T\floor{g(a)/T})d\tau\nonumber\\&&-\int_{g(b)\bmod T}^Tf(\tau)h'(\tau+T\floor{g(b)/T})d\tau
\end{eqnarray}

\begin{eqnarray}
\int_a^bf(\tau,g(\tau)\bmod T)d\tau&=&\int_{g(a)}^{g(b)}f(h(u),u\bmod T)h'(u)du\qquad\mbox{where }h=g^{-1}\\
&=&\sum_{k=0}^{\floor{g(b)/T}-\floor{g(a)/T}}\int_0^Tf(h(\tau+T(\floor{g(a)/T}+k),\tau)h'(\tau+T(\floor{g(a/T)}+k)d\tau\nonumber\\&&-\int_0^{g(a)\bmod T}f(h(\tau+T\floor{g(a)/T}),\tau)h'(\tau+T\floor{g(a)/T})d\tau\nonumber\\&&-\int_{g(b)\bmod T}^Tf(h(\tau+T\floor{g(b)/T}),\tau)h'(\tau+T\floor{g(b)/T})d\tau
\end{eqnarray}

\section{The concurrency condition}
There are many components to a particle system. But before even considering velocities, forces, and color gradients, there are 2 fundamental features of an $(R,L)$-emitter:
\begin{itemize}
\item $R(t),t\in[0,T)$: The spawn rate of the emitter as a function of time with a period $T$. Alternatively, we can extend its domain to $R(t\bmod T),t\in\mathbb{R}_+$. It must be true that for all $t\in[0,T)$, $R(t)\geq0$.
\item $L(t),t\in[0,T)$: The lifespan of a particle that would be spawned at $t$. Alternatively, we can extend its domain to $L(t\bmod T),t\in\mathbb{R}_+$. In addition, the lifespan can be reframed in terms of the particle index and not time, but this will be revisited later. It must be true that for all $t\in[0,T)$, $L(t)\geq0$. Also, particles spawned at a time $t$ where $L(t)=0$ are ignored.
\end{itemize}
If the particle system uses an immutable buffer or otherwise requires a maximum number of concurrent particles $M$, we need to be able bound the number of concurrent particles by $M$. This is formulated by
\begin{equation}N(t)=S(t)-D(t)\leq M,\quad\forall t\in\mathbb{R}_+\end{equation}
where $N(t)$ is the number of concurrent particles at time $t$, $S(t)$ is the total number of particles that have been spawned by time $t$, and $D(t)$ is the total number of particles that have been despawned by time $t$. We must therefore derive $S(t)$ and $D(t)$.

\subsection{Accumulated spawn amount}
Let's first derive $S(t)$. Note that:
\begin{equation}R=\frac{dS}{dt}\implies S(t)=\int_0^tR(\tau\bmod T)d\tau\end{equation}
For a multi-emission system, we can split $t$ into $\eta T+\sigma$, where $\eta=\floor{t/T}$ and $\sigma=t\bmod T$:
\begin{eqnarray}
\int_0^tR(\tau\bmod T)d\tau&=&\int_0^{\eta T}R(\tau\bmod T)d\tau+\int_0^\sigma R(\tau\bmod T)d\tau\\
&=&\eta\int_0^TR(\tau)d\tau+\int_0^\sigma R(\tau)d\tau\\
&=&\eta R_{\Sigma}+S(\sigma)
\end{eqnarray}
This spawn rate sum $R_\Sigma$ will be used throughout the theory:
\begin{equation}R_\Sigma=\int_0^TR(\tau)d\tau\end{equation}
Alternatively, we can use modular integration to decompose the integral:
\begin{eqnarray}
S(t)&=&\int_0^tR(\tau\bmod T)d\tau\\
&=&\int_{0\bmod T}^{t\bmod T}R(\tau)d\tau+(\floor{\tfrac{t}{T}}-\floor{\tfrac{0}{T}})\int_0^TR(\tau)d\tau\\
&=&\int_0^{t\bmod T}R(\tau)d\tau+\floor{\tfrac{t}{T}}R_\Sigma\\
&=&S(t\bmod T)+\floor{\tfrac{t}{T}}R_\Sigma
\end{eqnarray}
This is equivalent to the previous formulation, but will tie in better with the derivation of $N(t)$.

\subsection{Accumulated despawn amount}
Deriving $D(t)$ is trickier. Note that a particle must have spawned at some prior point in time $\tau$, and its lifespan must have expired ($\tau+L(\tau\bmod T)\leq t$):
\begin{equation}D(t)=\int_0^t\chi(\tau+L(\tau\bmod T)\leq t)R(\tau\bmod T)d\tau\end{equation}
where $\chi$ is the indicator function
\begin{equation}\chi(\beta)=\begin{cases}1&\mbox{if }\beta\mbox{ is true}\\0&\mbox{if }\beta\mbox{ is false}\end{cases}\end{equation}
Since the lifespan of a particle on one interval $[nT,(n+1)T)$ can expire on some later interval, there is overlapping and therefore we cannot simply split $t=\eta T+\sigma$ like we did before. Instead, we can use modular integration:
\begin{eqnarray}
D(t)&=&\int_0^t\chi(\tau+L(\tau\bmod T)\leq t)R(\tau\bmod T)d\tau\\
&=&\int_0^T\sum_{k=0}^{\floor{t/T}-\floor{0/T}}\chi(\tau+T(\floor{\tfrac{0}{T}}+k)+L(\tau)\leq t)R(\tau)d\tau\nonumber\\&&-\int_0^{0\bmod T}\chi(\tau+T\floor{\tfrac{0}{T}}+L(\tau)\leq t)R(\tau)d\tau\nonumber\\&&-\int_{t\bmod T}^T\chi(\tau+T\floor{\tfrac{t}{T}}+L(\tau)\leq t)R(\tau)d\tau\\
&=&\int_0^T\sum_{k=0}^{\floor{t/T}}\chi(\tau+L(\tau)\leq t-kT)R(\tau)d\tau-\int_{t\bmod T}^T\chi(\tau+L(\tau)\leq t\bmod T)R(\tau)d\tau
\end{eqnarray}

\subsection{Number of concurrent particles}
Now we can put the two accumulations together to compute the number of concurrent particles at any point in time:
\begin{eqnarray}
N(t)&=&S(t)-D(t)\\
&=&\int_0^{t\bmod T}R(\tau)d\tau+\floor{\tfrac{t}{T}}R_\Sigma\nonumber\\&&-\int_0^T\sum_{k=0}^{\floor{t/T}}\chi(\tau+L(\tau)\leq t-kT)R(\tau)d\tau+\int_{t\bmod T}^T\chi(\tau+L(\tau)\leq t\bmod T)R(\tau)d\tau\\
&=&\int_0^{t\bmod T}R(\tau)d\tau+\int_{t\bmod T}^T\chi(\tau+L(\tau)\leq t\bmod T)R(\tau)d\tau\nonumber\\&&-R_\Sigma+\sum_{k=0}^{\floor{t/T}}R_\Sigma-\int_0^T\sum_{k=0}^{\floor{t/T}}\chi(\tau+L(\tau)\leq t-kT)R(\tau)d\tau\\
&=&\int_0^{t\bmod T}R(\tau)d\tau-R_\Sigma+\int_{t\bmod T}^T\chi(\tau+L(\tau)\leq t\bmod T)R(\tau)d\tau\nonumber\\&&+\sum_{k=0}^{\floor{t/T}}\int_0^T(1-\chi(\tau+L(\tau)\leq t-kT))R(\tau)d\tau\\
&=&-\int_{t\bmod T}^T(1-\chi(\tau+L(\tau)\leq t\bmod T)R(\tau))d\tau\nonumber\\&&+\sum_{k=0}^{\floor{t/T}}\int_0^T(1-\chi(\tau+L(\tau)\leq t-kT))R(\tau)d\tau
\end{eqnarray}
Note that:
\begin{equation}1-\chi(\beta)=\chi(\neg\beta)\end{equation}
We continue:
\begin{eqnarray}
N(t)&=&\sum_{k=0}^{\floor{t/T}}\int_0^T\chi(\tau+L(\tau)>t-kT)R(\tau)d\tau-\int_{t\bmod T}^T\chi(\tau+L(\tau)>t\bmod T)R(\tau)d\tau
\end{eqnarray}
Note that the argument of the second indicator function is always true for $\tau\in[t\bmod T,T)$. This proves the following theorem.

\begin{thm}[Strong concurrency condition]
The number of concurrent particles of an $(R,L)$-emitter at any time $t\geq0$ is:
\begin{equation}\boxed{N(t)=\sum_{k=0}^{\floor{t/T}}\int_0^T\chi(\tau+L(\tau)>t-kT)R(\tau)d\tau-\int_{t\bmod T}^TR(\tau)d\tau}\end{equation}
Therefore, for a concurrency maximum $M$, the following condition must hold:
\begin{equation}\boxed{N_{\max}=\max_{t\in\mathbb{R}_+}\left(S(t\bmod T)+\sum_{k=0}^{\floor{t/T}}\int_0^T\chi(\tau+L(\tau)>t-kT)R(\tau)d\tau\right)-R_\Sigma\leq M}\end{equation}
\end{thm}

\begin{coro}[Weak concurrency condition]
Given a concurrency maximum $M$, any $(R,L)$-emitter must satisfy the following condition:
\begin{equation}\boxed{N_{\max}\leq M^*=R_\Sigma(1+L_{\max}/T)\leq M}\end{equation}
\begin{proof} First of all, $\chi(\tau)+L_{\max}>t-kT)$ is a generalization of $\chi(\tau+L(\tau)>t-kT)$. We can also independently take the maximum of the $S(t\bmod T)$ term (which is simply $R_\Sigma$):
\begin{eqnarray}
N_{\max}&\leq&R_\Sigma+\max_{t\in\mathbb{R_+}}\sum_{k=0}^{\floor{t/T}}\int_0^T\chi(\tau+L_{\max}>t-kT)R(\tau)d\tau-R_\Sigma\\
&=&\max_{t\in\mathbb{R_+}}\sum_{k=0}^{\floor{t/T}}\int_0^T\chi(\tau>t-L_{\max}-kT)R(\tau)d\tau
\end{eqnarray}
The condition in the indicator function is false for all $\tau\in[0,T]$ when $T\leq t-L_{\max}-kT\implies k\leq\floor{\frac{t-L_{\max}}{T}}-1$. So we can take the sum starting from $k=\floor{\frac{t-L_{\max}}{T}}$. We can also use this fact:
\begin{equation}\floor{\alpha}-\floor{\alpha-\beta}=\beta+(\alpha-\beta)\bmod1-\alpha\bmod1\leq\beta+1,\quad(\alpha,\beta\geq0)\end{equation}
So the maximum has this upper bound:
\begin{eqnarray}
N_{\max}&\leq&\max_{t\in\mathbb{R_+}}\sum_{k=\floor{(t-L_{\max})/T}}^{\floor{t/T}}\int_0^T\chi(\tau>t-L_{\max}-kT)R(\tau)d\tau\\
&\leq&\max_{t\in\mathbb{R_+}}\sum_{k=\floor{(t-L_{\max})/T}}^{\floor{t/T}}R_\Sigma\\
&\leq&\max_{t\in\mathbb{R_+}}(L_{\max}/T+1)R_\Sigma=M^*\leq M
\end{eqnarray}
\end{proof}
\end{coro}

\subsection{The particle domain conversion}
It may be more convenient to think of the lifespan function as a function of a particle's index $n$. Frame the corresponding particle-domain lifespan function as:
\begin{equation}L^*(n)\equiv L(t)\end{equation}
While $t$ may be continuous, $n$ is certainly discrete. If we denote the number of particles spawned in a period $T$ by
\begin{equation}\Phi=\floor{R_\Sigma}=\floor{\int_0^TR(\tau)d\tau}\end{equation}
then:
\begin{equation}t\in[0,T)\implies n\in[0,\Phi)\cap\mathbb{Z}\end{equation}
We can extend $t$ to $\mathbb{R}_+$ with:
\begin{equation}\eta_t=\floor{t/T}=\floor{n/\Phi}=\eta_n\end{equation}
and use the total particles spawned equation to derive the relationship between $t$ and $n$.  $n$ is easily seen as:
\begin{equation}n=\floor{S(t)}-1\end{equation}
Therefore, since $S$ can be computed, it can be used to determine $n$. Unfortunately, $S$ likely has no inverse due to periods of no spawning. However, we can use a similar function $\psi$ to the `inverse':
\begin{equation}t=\psi(n),\quad \psi(n)\sim S^{-1}(n+1)\end{equation}
More precisely, $t=\psi(n)$ is the smallest value of $t$ such that $S(t)=n+1$. In other words,
\begin{equation}\psi(n)=\min\left\{t\in\mathbb{R}_+:\;S(t)=n+1\right\}\end{equation}
For multi-emission systems, this can be written as:
\begin{equation}\psi(n)=\min\left\{t\in\mathbb{R}_+:\;\eta_t R_\Sigma+S(\sigma)=n+1\right\}\end{equation}
\begin{equation}\psi(n)=\eta_n T+\min\left\{\sigma\in[0,T):\;\eta_n R_\Sigma+S(\sigma)=n+1\right\}\end{equation}
\begin{equation}\psi(n)=\floor{n/\Phi}T+\min\left\{\sigma\in[0,T):\;S(\sigma)=n+1-\floor{n/\Phi}R_\Sigma\right\}\end{equation}
Thus, we can convert between time-domain and particle index-domain lifespan functions with:
\begin{equation}L(t)=L^*(\floor{S(t)}-1)\end{equation}
\begin{equation}L^*(n)=L(\psi(n))\end{equation}

\subsection{Spawn debt increments}
Define the spawn debt increment (increase in spawning amount after an update frame has passed) as:
\begin{equation}\mathcal{D}\{R\}=S(t)-S(t-\Delta t)=S(t\bmod T)-S((t-\Delta t)\bmod T)+(\floor{\tfrac{t}{T}}-\floor{\tfrac{t-\Delta t}{T}})R_\Sigma\end{equation}


\section{$(R,L)$-Emitter Examples}

\subsection{Constant spawn rate}
Let $R(t)=C$.
\begin{equation}S(t)=Ct\end{equation}
\begin{equation}R_\Sigma=CT,\qquad\Phi=\floor{CT}\end{equation}
\begin{equation}\psi(n)=\frac{n+1}{C}\end{equation}
\begin{equation}\mathcal{D}\{C\}=C(t\bmod T)-C((t-\Delta t)\bmod T)+(\floor{\tfrac{t}{T}}-\floor{\tfrac{t-\Delta t}{T}})CT=C\Delta t\end{equation}

\subsection{Constant lifespan}
Let $L(t)=L$.
\begin{eqnarray}
N(t)&=&\sum_{k=0}^{\floor{t/T}}\int_0^T\chi(\tau+L>t-kT)R(\tau)d\tau-\int_{t\bmod T}^TR(\tau)d\tau\\
&=&\sum_{k=0}^{\floor{t/T}}\int_{\clamp(t-kT-L,0,T)}^TR(\tau)d\tau-\int_{t\bmod T}^TR(\tau)d\tau
\end{eqnarray}
$$t-kT-L\in[0,T]\implies k\in\frac{1}{T}[t-L-T,t-L]$$
\begin{eqnarray}
N(t)&=&\sum_{k=\ceil{(t-L)/T}-1}^{\floor{(t-L)/T}}\chi(k\geq0)\int_{t-kT-L}^TR(\tau)d\tau-\int_{t\bmod T}^TR(\tau)d\tau\\
&=&\sum_{k=-\floor{(L-t)/T}-1}^{\floor{(t-L)/T}}\chi(k\geq0)\int_{t-kT-L}^TR(\tau)d\tau-\int_{t\bmod T}^TR(\tau)d\tau\\
&=&\chi(t\geq L)\int_{(t-L)\bmod T}^TR(\tau)d\tau-\int_{t\bmod T}^TR(\tau)d\tau
\end{eqnarray}

\subsection{Linear spawn rate}
Let $R(t)=(f-i)\tau/T+i$, where the spawn rate linearly interpolates between $(0,i)$ (inclusive) and $(T,f)$ (exclusive).
\begin{equation}\mathcal{D}\{R\}=i\Delta t+\frac{f-i}{2T}\left((t\bmod T)^2-((t-\Delta t)\bmod T)^2+T^2(\floor{\tfrac{t}{T}}-\floor{\tfrac{t-\Delta t}{T}})\right)\end{equation}

\subsection{Sinusoidal spawn rate}
Let $R(t)=a\sin(k\pi(t-b)/T)+c$, where $k$ is an integer.
\begin{equation}R_\Sigma=cT+\frac{2aT}{k\pi}(k\bmod 2)\cos(bk\pi/T)\end{equation}
\begin{equation}S(t)=ct-\frac{aT}{k\pi}\cos(k\pi(t-b)/T)\end{equation}
\begin{multline}\mathcal{D}\{R\}=c\Delta t-\frac{aT}{k\pi}\Big(\cos(k\pi(t\bmod T-b)/T)+\cos(k\pi((t-\Delta t)\bmod T-b)/T)\\+2(k\bmod 2)(\floor{\tfrac{t}{T}}-\floor{\tfrac{t-\Delta t}{T}})\cos(bk\pi/T)\Big)\end{multline}

\subsection{Discrete pulse spawn rate}
Let $R(t)=\sum_{t_i\in\mathcal{P}}\omega_i\delta(t-t_i)$.
\begin{equation}S(t)=\sum_{\substack{t_i\in\mathcal{P}\\t_i\leq t}}\omega_i,\qquad R_\Sigma=\sum_{t_i\in\mathcal{P}}\omega_i\end{equation}
\begin{equation}\mathcal{D}\{R\}=\sum_{t_i\in\mathcal{P}}\omega_i\Big(\chi(t_i\leq t\bmod T)-\chi(t_i\leq(t-\Delta t)\bmod T)\Big)+(\floor{\tfrac{t}{T}}-\floor{\tfrac{t-\Delta t}{T}})R_\Sigma\end{equation}
\begin{equation}\mathcal{D}\{R\}=\sum_{t_i\in\mathcal{P}}\omega_i\Big(\chi(t_i\leq t\bmod T)-\chi(t_i\leq(t-\Delta t)\bmod T)+\floor{\tfrac{t}{T}}-\floor{\tfrac{t-\Delta t}{T}}\Big)\end{equation}

\subsection{Continuous pulse spawn rate}
The continuous pulse emitter uses two `mirrored' power curves to finitize each pulse and add width. Given an integer weight $\omega_i$, powers $\alpha_i,\beta_i\geq0$, a middling point $t_i$, and two offset points $a_i\leq t_i,b_i\geq t_i,a_i\neq b_i$, it is defined as follows:
\begin{equation}R(t)=\sum_{i\in\mathcal{I}}R_i(t)\end{equation}
\begin{equation}R_i(t)=\frac{\omega_i}{\frac{t_i-a_i}{\alpha_i+1}+\frac{b_i-t_i}{\beta_i+1}}\cdot\begin{cases}\left(\frac{t-a_i}{t_i-a_i}\right)^{\alpha_i}&t\in[a_i,t_i]\\\left(\frac{t-b_i}{t_i-b_i}\right)^{\beta_i}&t\in[t_i,b_i]\\0&t\not\in[a_i,b_i]\end{cases}\end{equation}
The quotient in front gives it the characteristic of:
\begin{equation}\int_{a_i}^{b_i}R_i(\tau)d\tau=\omega_i\end{equation}
Let $m_i$ be the quotient for shorthand. In general,
\begin{eqnarray}
S(t)&=&\int_0^t\sum_{i\in\mathcal{I}}R_i(\tau)d\tau\\
&=&\sum_{i\in\mathcal{I}}\left(\int_{a_i}^{\clamp(t,a_i,t_i)}R_i(\tau)d\tau+\int_{t_i}^{\clamp(t,t_i,b_i)}R_i(\tau)d\tau\right)\\
&=&m_i\sum_{i\in\mathcal{I}}\left(\int_{a_i}^{\clamp(t,a_i,t_i)}\left(\frac{\tau-a_i}{t_i-a_i}\right)^{\alpha_i}d\tau+\int_{t_i}^{\clamp(t,t_i,b_i)}\left(\frac{\tau-b_i}{t_i-b_i}\right)^{\beta_i}d\tau\right)\\
&=&m_i\sum_{i\in\mathcal{I}}\left(\frac{1}{(t_i-a_i)^{\alpha_i}}\int_{a_i}^{\clamp(t,a_i,t_i)}(\tau-a_i)^{\alpha_i}d\tau+\frac{1}{(t_i-b_i)^{\beta_i}}\int_{t_i}^{\clamp(t,t_i,b_i)}(\tau-b_i)^{\beta_i}d\tau\right)\\
&=&m_i\sum_{i\in\mathcal{I}}\left(\frac{(\tau-a_i)^{\alpha_i+1}}{(\alpha_i+1)(t_i-a_i)^{\alpha_i}}\Big|_{a_i}^{\clamp(t,a_i,t_i)}+\frac{(\tau-b_i)^{\beta_i+1}}{(\beta_i+1)(t_i-b_i)^{\beta_i}}\Big|_{t_i}^{\clamp(t,t_i,b_i)}\right)\\
&=&\sum_{i\in\mathcal{I}}\begin{cases}
0&t\leq a_i\\
\frac{m_i(t-a_i)^{\alpha_i+1}}{(\alpha_i+1)(t_i-a_i)^{\alpha_i}}&t\in(a_i,t_i]\\
\omega_i-\frac{m_i(b_i-t)^{\beta_i+1}}{(\beta_i+1)(b_i-t_i)^{\beta_i}}&t\in[t_i,b_i)\\
\omega_i&t\geq b_i
\end{cases}
\end{eqnarray}
There is no simplification for this in the spawn debt increment formula, so use $S(t)$ as is. The only exception is for $\alpha_0=\beta_i=0$, where $m_i=\frac{\omega_i}{b_i-a_i}$:
\begin{eqnarray}
S(t)&=&\sum_{i\in\mathcal{I}}\begin{cases}
0&t\leq a_i\\
\frac{\omega_i}{b_i-a_i}(t-a_i)&t\in(a_i,t_i]\\
\omega_i-\frac{\omega_i}{b_i-a_i}(b_i-t)&t\in[t_i,b_i)\\
\omega_i&t\geq b_i
\end{cases}\\
&=&\sum_{i\in\mathcal{I}}\begin{cases}
0&t\leq a_i\\
\frac{\omega_i(t-a_i)}{b_i-a_i}&t\in(a_i,b_i)\\
\omega_i&t\geq b_i
\end{cases}
\end{eqnarray}
\begin{eqnarray}
\mathcal{D}\{R\}&=&(\floor{\tfrac{t}{T}}-\floor{\tfrac{t-\Delta t}{T}})R_\Sigma+
\sum_{i\in\mathcal{I}}\begin{cases}
0&t\bmod T\leq a_i\\
\frac{\omega_i}{b_i-a_i}(t\bmod T)-\frac{a_i\omega_i}{b_i-a_i}&t\bmod T\in(a_i,b_i)\\
\omega_i&t\bmod T\geq b_i
\end{cases}\nonumber\\&&
-\sum_{i\in\mathcal{I}}\begin{cases}
0&(t-\Delta t)\bmod T\leq a_i\\
\frac{\omega_i}{b_i-a_i}((t-\Delta t)\bmod T)-\frac{a_i\omega_i}{b_i-a_i}&(t-\Delta t)\bmod T\in(a_i,b_i)\\
\omega_i&(t-\Delta t)\bmod T\geq b_i
\end{cases}\\
&=&(\floor{\tfrac{t}{T}}-\floor{\tfrac{t-\Delta t}{T}})R_\Sigma+\sum_{i\in\mathcal{I}}\frac{\omega_i}{b_i-a_i}(\clamp(t\bmod T,a_i,b_i)-\clamp((t-\Delta t)\bmod T), a_i, b_i))\qquad\qquad\\
&=&\sum_{i\in\mathcal{I}}\omega_i\Big(\floor{\tfrac{t}{T}}-\floor{\tfrac{t-\Delta t}{T}}+\frac{1}{b_i-a_i}(\clamp(t\bmod T,a_i,b_i)-\clamp((t-\Delta t)\bmod T), a_i, b_i))\Big)\qquad\qquad
\end{eqnarray}

\section{Random distributions}
Given a 1D probability density function (PDF) $f$, the 1D cumulative density function (CDF) is defined as:
\begin{equation}F(x)=\int_{-\infty}^xf(u)du\end{equation}
The 1D RNG transformer which takes a randomly generated number on $[0,1]$ and transforms it according to the PDF is defined as:
\begin{equation}T(r)\sim F^{-1}(r),\qquad T(r)=\chs\{x\in\mathbb{R}:\;F(x)=r\}\end{equation}
The corresponding 2D and 3D definitions are:
\begin{equation}F(x,y)=\int_{-\infty}^y\int_{-\infty}^xf(u,v)dudv,\qquad T(r_1,r_2)=\chs\{(x,y)\in\mathbb{R}^2:\;F(x,y)=(r_1,r_2)\}\end{equation}
\begin{equation}F(x,y,z)=\int_{-\infty}^z\int_{-\infty}^y\int_{-\infty}^xf(u,v,w)dudvdw,\qquad T(r_1,r_2, r_3)=\chs\{(x,y,z)\in\mathbb{R}^3:\;F(x,y,z)=(r_1,r_2,r_3)\}\end{equation}
A cutoff bound can always be used to clamp the output of the transformer. Additionally, all random distributions in this section have mean = 0, and generate an offset to the mean. This way, transforming the range via translation/rotation/scaling can be done independently.

\subsection{Uniform distribution}
\subsubsection{1-dimensional}
Parameters:
\begin{itemize}
\item $d\in\mathbb{R}_{>0}$: offset length.
\end{itemize}
\begin{equation}f(x)=\frac{1}{2d}\chi(x\in[-d,d])\end{equation}
\begin{equation}F(x)=\int_{-\infty}^x\frac{1}{2d}\chi(u\in[-d,d])du=\begin{cases}
0&x<-d\\
\frac{1}{2d}(x+d)&x\in[-d,d]\\
1&x>d
\end{cases}\end{equation}
\begin{equation}T(r)=2dr-d\end{equation}
\subsubsection{2-dimensional}
Parameters:
\begin{itemize}
\item $d_x,d_y\in\mathbb{R}_{>0}$: offset lengths.
\end{itemize}
\begin{equation}f(x,y)=\frac{1}{4d_xd_y}\chi(x\in[-d_x,d_x])\chi(y\in[-d_y,d_y])\end{equation}
\begin{multline}F(x,y)=\int_{-\infty}^y\int_{-\infty}^x\frac{1}{4d_xd_y}\chi(u\in[-d_x,d_x])\chi(v\in[-d_y,d_y])dudv\\=\begin{cases}
0&x<-d_x\mbox{ or }y<-d_y\\
\frac{1}{4d_xd_y}(x+d_x)(y+d_y)&x\in[-d_x,d_x]\mbox{ and }y\in[-d_y,d_y]\\
\frac{1}{2d_x}(x+d_x)&x\in[-d_x,d_x]\mbox{ and }y>d_y\\
\frac{1}{2d_y}(y+d_y)&x>d_x\mbox{ and }y\in[-d_y,d_y]\\
1&x>d_x\mbox{ and }y>d_y
\end{cases}\end{multline}
\begin{equation}T(r_1,r_2)=(2d_xr_1-d_x,2d_yr_2-d_y)\end{equation}
\subsubsection{3-dimensional}
By extension,
\begin{equation}T(r_1,r_2,r_3)=(2d_xr_1-d_x,2d_yr_2-d_y,2d_zr_3-d_z)\end{equation}

\subsection{Logistic-Bell distribution}
\subsubsection{1-dimensional}
Parameters:
\begin{itemize}
\item $\mu\in\mathbb{R}$: peak position.
\item $h\in\mathbb{R}_{>0}$: height of curve.
\end{itemize}
\begin{equation}f(x)=\frac{4he^{-4h(x-\mu)}}{(1+e^{-4h(x-\mu)})^2}\end{equation}
\begin{equation}F(x)=\int_{-\infty}^x\frac{4he^{-4h(u-\mu)}}{(1+e^{-4h(u-\mu)})^2}du=\frac{1}{1+e^{-4h(x-\mu)}}\end{equation}
\begin{equation}T(r)=\mu-\frac{1}{4h}\ln\left(\frac{1}{r}-1\right)\end{equation}
A safer approach would be:
\begin{equation}T(r)=\begin{cases}
\mu+\frac{1}{4h}\ln\left(\frac{r}{1-r}\right)&r\in(\delta_1,\delta_2)\\
\mu+\frac{1}{4h}\ln\left(\frac{\delta_1}{1-\delta_1}\right)&r<\delta_1\\
\mu+\frac{1}{4h}\ln\left(\frac{\delta_2}{1-\delta_2}\right)&r>\delta_2
\end{cases}\end{equation}
or more simply:
\begin{equation}T(r)=\begin{cases}
\mu+\frac{1}{4h}\ln\left(\frac{r}{1-r}\right)&r\in(\delta_1,\delta_2)\\
\mu-d_1&r<\delta_1\\
\mu+d_2&r>\delta_2
\end{cases}\end{equation}
\subsubsection{2-dimensional}
There is no simple extension to 2D, but another approach can be used where the bell curve is rotationally symmetric. In that case, the radius $\rho$ is determined by one half-bell curve and the angle $\theta$ is uniform over $[0,2\pi)$:
\begin{eqnarray}T(r_1,r_2)=(\rho,\theta)=(\tfrac{1}{2h}\tanh^{-1}(r_1),2\pi r_2)\end{eqnarray}
\begin{eqnarray}\implies T(r_1,r_2)=(x,y)=(\rho\cos\theta,\rho\sin\theta)=\left(\tfrac{1}{2h}\tanh^{-1}(r_1)\cos(2\pi r_2),\tfrac{1}{2h}\tanh^{-1}(r_1)\sin(2\pi r_2)\right)\end{eqnarray}
\subsubsection{3-dimensional}
This is similar to the 2D case, but spherical coordinates are used instead:
\begin{multline}T(r_1,r_2,r_2)=(x,y,z)=(\rho\cos\theta\sin\phi,\rho\sin\theta\sin\phi,\rho\cos\phi)\\=\left(\tfrac{1}{2h}\tanh^{-1}(r_1)\cos(2\pi r_2)\sin(\pi r_3),\tfrac{1}{2h}\tanh^{-1}(r_1)\sin(2\pi r_2)\sin(\pi r_3),\tfrac{1}{2h}\tanh^{-1}(r_1)\cos(\pi r_3)\right)\end{multline}



\end{document}